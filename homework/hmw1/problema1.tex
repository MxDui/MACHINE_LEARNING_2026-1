\section*{1. [2.5 puntos] Dimensión VC}

Para una clase de hipótesis de triángulos $\mathcal{H}_{\text{tri}}$:

\begin{enumerate}
    \item[a)] Escribe una función indicadora para evaluar si un punto 2D pertenece o no a una hipótesis de un triángulo particular $h_{\boldsymbol{\theta}} \in \mathcal{H}_{\text{tri}}$.
    
    \textbf{Solución:}
    
    Usando coordenadas baricéntricas:
    
    $$h_{\boldsymbol{\theta}}(x,y) = \begin{cases}
    1 & \text{si } \lambda_1 \geq 0, \lambda_2 \geq 0, \lambda_3 \geq 0 \\
    0 & \text{en caso contrario}
    \end{cases}$$
    
    donde:
    
    $$\lambda_1 = \frac{(y_2 - y_3)(x - x_3) + (x_3 - x_2)(y - y_3)}{(y_2 - y_3)(x_1 - x_3) + (x_3 - x_2)(y_1 - y_3)}$$
    
    $$\lambda_2 = \frac{(y_3 - y_1)(x - x_3) + (x_1 - x_3)(y - y_3)}{(y_2 - y_3)(x_1 - x_3) + (x_3 - x_2)(y_1 - y_3)}$$
    
    $$\lambda_3 = 1 - \lambda_1 - \lambda_2$$
    
    % HINT: La función indicadora debe verificar si un punto (x,y) está dentro del triángulo.
    % Puedes usar el método de coordenadas baricéntricas o el método del área.
    % 
    % SOLUCIÓN SUGERIDA:
    % - Parametriza el triángulo con 6 parámetros: theta = (x1, y1, x2, y2, x3, y3)
    % - Función indicadora usando áreas:
    %   h_theta(x,y) = 1 si el punto está dentro, 0 si está fuera
    % - Método del área: Un punto está dentro si:
    %   Area(ABC) = Area(PBC) + Area(APC) + Area(ABP)
    % - Alternativamente, usa coordenadas baricéntricas:
    %   El punto P está dentro si todas las coordenadas baricéntricas son >= 0
    
    \item[b)] Debes escribir la función parametrizada por los parámetros $\boldsymbol{\theta}$ que tú consideres.
    
    \textbf{Solución:}
    
    $$\boldsymbol{\theta} = (x_1, y_1, x_2, y_2, x_3, y_3) \in \mathbb{R}^6$$
    
    Función parametrizada:
    
    $$h_{\boldsymbol{\theta}}(x,y) = \begin{cases}
    1 & \text{si el punto } (x,y) \text{ está dentro del triángulo} \\
    0 & \text{en caso contrario}
    \end{cases}$$
    
    Implementación:
    
    \begin{verbatim}
def indicator_triangle(x, y, theta):
    x1, y1, x2, y2, x3, y3 = theta
    denom = (y2 - y3) * (x1 - x3) + (x3 - x2) * (y1 - y3)
    
    if abs(denom) < 1e-10:
        return 0
    
    lambda1 = ((y2 - y3) * (x - x3) + (x3 - x2) * (y - y3)) / denom
    lambda2 = ((y3 - y1) * (x - x3) + (x1 - x3) * (y - y3)) / denom
    lambda3 = 1 - lambda1 - lambda2
    
    return 1 if (lambda1 >= 0 and lambda2 >= 0 and lambda3 >= 0) else 0
    \end{verbatim}
    
    % HINT: Los parámetros más naturales son los 3 vértices del triángulo.
    % theta = {(x1,y1), (x2,y2), (x3,y3)} - 6 parámetros en total
    % 
    % EJEMPLO DE FUNCIÓN:
    % def indicator_triangle(x, y, theta):
    %     x1, y1, x2, y2, x3, y3 = theta
    %     # Calcular usando el método de tu elección
    %     # Retornar 1 si está dentro, 0 si está fuera
    
    \item[c)] Luego usarás dicha función en la práctica asociada a este ejercicio para demostrar exhaustivamente (mediante un programa en Python) que la dimensión VC de la clase $\mathcal{H}_{\text{tri}}$ de triángulos es 7.
    
    \textbf{Solución:}
    
    La dimensión VC es 7:
    
    \textbf{1. Existen 7 puntos que pueden ser shattered:}
    
    Toma 7 puntos en un círculo: $P_k = (\cos(2\pi k/7), \sin(2\pi k/7))$ para $k=0,1,\dots,6$.
    
    Un triángulo es la intersección de 3 semiplanos, por lo que puede crear hasta 3 intervalos de puntos positivos sobre el círculo. Con 7 puntos, cualquier etiquetado tiene a lo más 6 cambios, realizable con 3 intervalos.
    
    \textbf{2. NO existen 8 puntos que puedan ser shattered:}
    
    Con 8 puntos en círculo y etiquetado alternante (+, -, +, -, ...), hay 8 transiciones que requieren 4 intervalos. Un triángulo solo puede generar 3 intervalos, por lo que este etiquetado no es realizable.
    
    Implementación:
    \begin{verbatim}
    # Verificar que 7 puntos pueden ser shattered
    puntos_7 = generar_7_puntos_en_circulo()
    for etiquetado in all_labelings(2**7):
        assert existe_triangulo_separador(puntos_7, etiquetado)
    
    # Verificar que 8 puntos NO pueden ser shattered
    puntos_8 = generar_8_puntos_en_circulo()
    etiquetado_alt = alternar_signos(8)  # + - + - + - + -
    assert not existe_triangulo_separador(puntos_8, etiquetado_alt)
    \end{verbatim}
    
    Por tanto, VC-dim = 7.
    
    % HINT: Para demostrar VC-dim = 7:
    % 1. Encuentra 7 puntos que pueden ser shattered (separados de todas las 2^7 formas posibles)
    % 2. Demuestra que NO existen 8 puntos que puedan ser shattered
    % 
    % ESTRATEGIA PYTHON:
    % - Genera conjuntos de 7 puntos y verifica que pueden ser shattered
    % - Para cada etiquetado posible (2^7 = 128), encuentra un triángulo que lo separe
    % - Prueba con conjuntos de 8 puntos y muestra que al menos un etiquetado no es separable
    % - Sugerencia: Usa puntos en configuración especial (e.g., hexágono regular + centro)
\end{enumerate}
