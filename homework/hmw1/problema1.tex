\section*{1. [2.5 puntos] Dimensión VC}

Para una clase de hipótesis de triángulos $\mathcal{H}_{\text{tri}}$:

\begin{enumerate}
    \item[a)] Escribe una función indicadora para evaluar si un punto 2D pertenece o no a una hipótesis de un triángulo particular $h_{\boldsymbol{\theta}} \in \mathcal{H}_{\text{tri}}$.
        
    La función indicadora debe determinar si un punto $(x,y) \in \mathbb{R}^2$ está dentro o fuera de un triángulo, regresando 1 si el punto está dentro y 0 si está fuera.
        
    Utilizo las coordenadas baricéntricas para resolver este problema. Para un triángulo con vértices $V_1 = (x_1, y_1)$, $V_2 = (x_2, y_2)$, $V_3 = (x_3, y_3)$, cualquier punto $P = (x, y)$ se puede expresar como:
    
    $$P = \lambda_1 V_1 + \lambda_2 V_2 + \lambda_3 V_3$$
    
    donde $\lambda_1 + \lambda_2 + \lambda_3 = 1$.
    
    Las coordenadas $\lambda_1$, $\lambda_2$ y $\lambda_3$ se obtienen resolviendo el sistema:
    
    \begin{align}
    x &= \lambda_1 x_1 + \lambda_2 x_2 + \lambda_3 x_3 \\
    y &= \lambda_1 y_1 + \lambda_2 y_2 + \lambda_3 y_3 \\
    1 &= \lambda_1 + \lambda_2 + \lambda_3
    \end{align}
    
    La solución es:
    
    $$\lambda_1 = \frac{(y_2 - y_3)(x - x_3) + (x_3 - x_2)(y - y_3)}{(y_2 - y_3)(x_1 - x_3) + (x_3 - x_2)(y_1 - y_3)}$$
    
    $$\lambda_2 = \frac{(y_3 - y_1)(x - x_3) + (x_1 - x_3)(y - y_3)}{(y_2 - y_3)(x_1 - x_3) + (x_3 - x_2)(y_1 - y_3)}$$
    
    $$\lambda_3 = 1 - \lambda_1 - \lambda_2$$
    
    \textbf{Criterio de pertenencia:}
    
    Un punto pertenece al triángulo si y solo si todas sus coordenadas baricéntricas son no negativas:
    
    $$P \in \text{Triángulo} \Leftrightarrow \lambda_1 \geq 0 \land \lambda_2 \geq 0 \land \lambda_3 \geq 0$$
    
    \textbf{Función indicadora resultante:}
    
    $$h_{\boldsymbol{\theta}}(x,y) = \begin{cases}
    1 & \text{si } \lambda_1 \geq 0 \land \lambda_2 \geq 0 \land \lambda_3 \geq 0 \\
    0 & \text{en caso contrario}
    \end{cases}$$
    
    
    \item[b)] Debes escribir la función parametrizada por los parámetros $\boldsymbol{\theta}$ que tú consideres.
        
    \textbf{Parametrización propuesta:}
    
    Un triángulo en $\mathbb{R}^2$ queda completamente determinado por sus tres vértices. Cada vértice tiene dos coordenadas $(x_i, y_i)$, por lo que defino 6 parámetros en total.
    
    \textbf{Vector de parámetros:}
    
    $$\boldsymbol{\theta} = (x_1, y_1, x_2, y_2, x_3, y_3) \in \mathbb{R}^6$$
    
    \textbf{Función parametrizada:}
    
    La función $h_{\boldsymbol{\theta}}: \mathbb{R}^2 \rightarrow \{0, 1\}$ se define como:
    
    $$h_{\boldsymbol{\theta}}(x,y) = \mathbb{1}_{\{\lambda_1(\boldsymbol{\theta}, x, y) \geq 0\}} \cdot \mathbb{1}_{\{\lambda_2(\boldsymbol{\theta}, x, y) \geq 0\}} \cdot \mathbb{1}_{\{\lambda_3(\boldsymbol{\theta}, x, y) \geq 0\}}$$
    
    donde $\mathbb{1}_{\{A\}}$ es la función indicadora del evento $A$.
    
    \textbf{Implementación explícita:}
    
    Definiendo el denominador común:
    $$D(\boldsymbol{\theta}) = (y_2 - y_3)(x_1 - x_3) + (x_3 - x_2)(y_1 - y_3)$$
    
    Si $|D(\boldsymbol{\theta})| > \epsilon$ (triángulo no degenerado), entonces:
    
    $$h_{\boldsymbol{\theta}}(x,y) = \begin{cases}
    1 & \text{si } \begin{aligned}
    &\frac{(y_2 - y_3)(x - x_3) + (x_3 - x_2)(y - y_3)}{D(\boldsymbol{\theta})} \geq 0 \text{ y} \\
    &\frac{(y_3 - y_1)(x - x_3) + (x_1 - x_3)(y - y_3)}{D(\boldsymbol{\theta})} \geq 0 \text{ y} \\
    &1 - \lambda_1 - \lambda_2 \geq 0
    \end{aligned} \\
    0 & \text{en caso contrario}
    \end{cases}$$
    
    Si $|D(\boldsymbol{\theta})| \leq \epsilon$, entonces $h_{\boldsymbol{\theta}}(x,y) = 0$ (triángulo degenerado).
  
\end{enumerate}