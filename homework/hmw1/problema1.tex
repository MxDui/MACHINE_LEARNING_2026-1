\section*{1. [2.5 puntos] Dimensión VC}

Para una clase de hipótesis de triángulos $\mathcal{H}_{\text{tri}}$:

\begin{enumerate}
    \item[a)] Escribe una función indicadora para evaluar si un punto 2D pertenece o no a una hipótesis de un triángulo particular $h_{\boldsymbol{\theta}} \in \mathcal{H}_{\text{tri}}$.
    
    \textbf{Solución:}
    
    La función indicadora para determinar si un punto $(x,y)$ pertenece a un triángulo con vértices en $(x_1, y_1)$, $(x_2, y_2)$, $(x_3, y_3)$ se puede definir utilizando \textbf{coordenadas baricéntricas}:
    
    $$h_{\boldsymbol{\theta}}(x,y) = \begin{cases}
    1 & \text{si } \lambda_1 \geq 0, \lambda_2 \geq 0, \lambda_3 \geq 0 \\
    0 & \text{en caso contrario}
    \end{cases}$$
    
    donde las coordenadas baricéntricas $\lambda_1$, $\lambda_2$, $\lambda_3$ se calculan como:
    
    $$\lambda_1 = \frac{(y_2 - y_3)(x - x_3) + (x_3 - x_2)(y - y_3)}{(y_2 - y_3)(x_1 - x_3) + (x_3 - x_2)(y_1 - y_3)}$$
    
    $$\lambda_2 = \frac{(y_3 - y_1)(x - x_3) + (x_1 - x_3)(y - y_3)}{(y_2 - y_3)(x_1 - x_3) + (x_3 - x_2)(y_1 - y_3)}$$
    
    $$\lambda_3 = 1 - \lambda_1 - \lambda_2$$
    
    \textbf{Interpretación geométrica:} Un punto está dentro del triángulo si y solo si todas sus coordenadas baricéntricas son no negativas. Esto equivale a que el punto puede expresarse como una combinación convexa de los tres vértices.
    
    \textbf{Método alternativo (usando áreas):}
    
    $$h_{\boldsymbol{\theta}}(x,y) = \begin{cases}
    1 & \text{si } |A_{ABC}| = |A_{PBC}| + |A_{APC}| + |A_{ABP}| \\
    0 & \text{en caso contrario}
    \end{cases}$$
    
    donde $A_{ABC}$ es el área del triángulo con vértices $A=(x_1,y_1)$, $B=(x_2,y_2)$, $C=(x_3,y_3)$, y $P=(x,y)$ es el punto a evaluar.
    
    % HINT: La función indicadora debe verificar si un punto (x,y) está dentro del triángulo.
    % Puedes usar el método de coordenadas baricéntricas o el método del área.
    % 
    % SOLUCIÓN SUGERIDA:
    % - Parametriza el triángulo con 6 parámetros: θ = (x1, y1, x2, y2, x3, y3)
    % - Función indicadora usando áreas:
    %   h_θ(x,y) = 1 si el punto está dentro, 0 si está fuera
    % - Método del área: Un punto está dentro si:
    %   Area(ABC) = Area(PBC) + Area(APC) + Area(ABP)
    % - Alternativamente, usa coordenadas baricéntricas:
    %   El punto P está dentro si todas las coordenadas baricéntricas son >= 0
    
    \item[b)] Debes escribir la función parametrizada por los parámetros $\boldsymbol{\theta}$ que tú consideres.
    
    \textbf{Solución:}
    
    La parametrización natural para un triángulo en $\mathbb{R}^2$ requiere \textbf{6 parámetros}:
    
    $$\boldsymbol{\theta} = (x_1, y_1, x_2, y_2, x_3, y_3) \in \mathbb{R}^6$$
    
    donde $(x_i, y_i)$ representa las coordenadas del $i$-ésimo vértice del triángulo.
    
    La función parametrizada es entonces:
    
    $$h_{\boldsymbol{\theta}}: \mathbb{R}^2 \rightarrow \{0,1\}$$
    
    $$h_{\boldsymbol{\theta}}(x,y) = \mathbb{1}_{\{\text{punto } (x,y) \text{ está dentro del triángulo definido por } \boldsymbol{\theta}\}}$$
    
    \textbf{Expresión explícita:}
    
    $$h_{\boldsymbol{\theta}}(x,y) = \begin{cases}
    1 & \text{si } \begin{aligned}
    &\text{sgn}((x-x_1)(y_2-y_1) - (y-y_1)(x_2-x_1)) = \\
    &\text{sgn}((x-x_2)(y_3-y_2) - (y-y_2)(x_3-x_2)) = \\
    &\text{sgn}((x-x_3)(y_1-y_3) - (y-y_3)(x_1-x_3))
    \end{aligned} \\
    0 & \text{en caso contrario}
    \end{cases}$$
    
    donde $\text{sgn}$ es la función signo. Esta formulación verifica que el punto esté del mismo lado de las tres aristas del triángulo.
    
    \textbf{Implementación en Python:}
    
    \begin{verbatim}
def indicator_triangle(x, y, theta):
    x1, y1, x2, y2, x3, y3 = theta
    
    # Calcular el denominador común
    denominator = (y2 - y3) * (x1 - x3) + (x3 - x2) * (y1 - y3)
    
    # Si el denominador es cero, el triángulo es degenerado
    if abs(denominator) < 1e-10:
        return 0
    
    # Calcular coordenadas baricéntricas
    lambda1 = ((y2 - y3) * (x - x3) + (x3 - x2) * (y - y3)) / denominator
    lambda2 = ((y3 - y1) * (x - x3) + (x1 - x3) * (y - y3)) / denominator
    lambda3 = 1 - lambda1 - lambda2
    
    # Verificar si el punto está dentro del triángulo
    return 1 if (lambda1 >= 0 and lambda2 >= 0 and lambda3 >= 0) else 0
    \end{verbatim}
    
    \textbf{Observación:} La clase de hipótesis de triángulos es:
    
    $$\mathcal{H}_{\text{tri}} = \{h_{\boldsymbol{\theta}} : \boldsymbol{\theta} \in \mathbb{R}^6 \text{ define un triángulo no degenerado}\}$$
    
    Un triángulo es no degenerado si $|(y_2 - y_3)(x_1 - x_3) + (x_3 - x_2)(y_1 - y_3)| > 0$, es decir, los tres vértices no son colineales.
    
    % HINT: Los parámetros más naturales son los 3 vértices del triángulo.
    % theta = {(x1,y1), (x2,y2), (x3,y3)} - 6 parámetros en total
    % 
    % EJEMPLO DE FUNCIÓN:
    % def indicator_triangle(x, y, theta):
    %     x1, y1, x2, y2, x3, y3 = theta
    %     # Calcular usando el método de tu elección
    %     # Retornar 1 si está dentro, 0 si está fuera
    
    \item[c)] Luego usarás dicha función en la práctica asociada a este ejercicio para demostrar exhaustivamente (mediante un programa en Python) que la dimensión VC de la clase $\mathcal{H}_{\text{tri}}$ de triángulos es 7.
    
    \textbf{Solución:}
    
    Para demostrar que la dimensión VC de los triángulos es 7, se debe:
    
    \textbf{1. Demostrar que existen 7 puntos que pueden ser ``shattered'':}
    
    Una configuración que funciona es usar 6 puntos en los vértices de un hexágono regular y un punto en el centro:
    \begin{itemize}
        \item Puntos del hexágono: $(\cos(k\pi/3), \sin(k\pi/3))$ para $k = 0, 1, 2, 3, 4, 5$
        \item Punto central: $(0, 0)$
    \end{itemize}
    
    \textbf{2. Demostrar que NO existen 8 puntos que puedan ser shattered:}
    
    Para cualquier conjunto de 8 puntos en posición general, siempre existe al menos un etiquetado de las $2^8 = 256$ posibles formas que no puede ser realizado por ningún triángulo. Esto se debe a que:
    
    \begin{itemize}
        \item Un triángulo divide el plano en exactamente 2 regiones (interior y exterior)
        \item Con 8 puntos, existen etiquetados que requieren regiones no conexas para la clase positiva
        \item Por ejemplo, si 4 puntos forman un cuadrilátero convexo y los otros 4 están dentro, no se puede separar solo los puntos internos sin incluir algunos externos
    \end{itemize}
    
    \textbf{Pseudocódigo para la implementación:}
    \begin{verbatim}
    # Verificar que 7 puntos pueden ser shattered
    puntos_7 = generar_configuracion_hexagono_centro()
    for etiquetado in all_labelings(2^7):
        assert existe_triangulo_separador(puntos_7, etiquetado)
    
    # Verificar que 8 puntos NO pueden ser shattered
    puntos_8 = generar_8_puntos_posicion_general()
    contador_no_separables = 0
    for etiquetado in all_labelings(2^8):
        if not existe_triangulo_separador(puntos_8, etiquetado):
            contador_no_separables += 1
    assert contador_no_separables > 0
    \end{verbatim}
    
    Por tanto, la dimensión VC de la clase de hipótesis de triángulos es exactamente 7.
    
    % HINT: Para demostrar VC-dim = 7:
    % 1. Encuentra 7 puntos que pueden ser shattered (separados de todas las 2^7 formas posibles)
    % 2. Demuestra que NO existen 8 puntos que puedan ser shattered
    % 
    % ESTRATEGIA PYTHON:
    % - Genera conjuntos de 7 puntos y verifica que pueden ser shattered
    % - Para cada etiquetado posible (2^7 = 128), encuentra un triángulo que lo separe
    % - Prueba con conjuntos de 8 puntos y muestra que al menos un etiquetado no es separable
    % - Sugerencia: Usa puntos en configuración especial (e.g., hexágono regular + centro)
\end{enumerate}
