\section*{4. [2.5 puntos] Función de pérdida relajada}

En los ejercicios 1 y 2 obtuviste funciones indicadoras para evaluar la pertenencia o no de puntos a hipótesis concretas.

\begin{enumerate}
    \item[a)] Debes extender la función indicadora para crear una nueva función que no evalúe de manera estrictamente ``dura'' la pertenencia a la hipótesis sino que permita considerar puntos cercanos a la frontera de la hipótesis de manera penalizada.
    
    \textbf{Solución:}
    
    Función de pérdida suave para la elipse:
    
    $$L_{\text{soft}}(x, y) = \begin{cases}
    0 & \text{si } d(x,y) \leq 1 \\
    \frac{d(x,y) - 1}{\delta} & \text{si } 1 < d(x,y) \leq 1 + \delta \\
    1 & \text{si } d(x,y) > 1 + \delta
    \end{cases}$$
    
    donde $d(x,y) = \sqrt{\frac{(x-h)^2}{a^2} + \frac{(y-k)^2}{b^2}}$ y $\delta > 0$ controla el ancho de transición.
    
    
    \item[b)] Tú debes definir una manera de extender la frontera hasta cierta distancia máxima más allá de la frontera original. Por ejemplo: englobando a la hipótesis dentro de un círculo con cierto radio, o definiendo una distancia $\Delta d$ más allá de la frontera de la hipótesis.
    
    \textbf{Solución:}
    
    Zona de transición con ancho $\delta$:
    
    \begin{itemize}
        \item Frontera original: $d(x,y) = 1$
        \item Zona de transición: $1 < d(x,y) \leq 1 + \delta$
        \item Región exterior: $d(x,y) > 1 + \delta$
    \end{itemize}
    
    La zona forma una cáscara elíptica entre:
    \begin{itemize}
        \item Elipse original: $\frac{(x-h)^2}{a^2} + \frac{(y-k)^2}{b^2} = 1$
        \item Elipse expandida: $\frac{(x-h)^2}{(a(1+\delta))^2} + \frac{(y-k)^2}{(b(1+\delta))^2} = 1$
    \end{itemize}
    
    
    \item[c)] En la región donde extiendas la frontera la función de pérdida debe penalizar en lugar de evaluar a cero.
    
    \textbf{Solución:}
    
    $$L_{\text{relajada}}(x, y) = \begin{cases}
    0 & \text{si } d(x,y) \leq 1 \\
    \frac{d(x,y) - 1}{\delta} & \text{si } 1 < d(x,y) \leq 1 + \delta \\
    1 & \text{si } d(x,y) > 1 + \delta
    \end{cases}$$
    
    Penalización:
    \begin{itemize}
        \item Interior ($d \leq 1$): Sin pérdida
        \item Transición ($1 < d \leq 1 + \delta$): Penalización lineal de 0 a 1
        \item Exterior ($d > 1 + \delta$): Pérdida máxima
    \end{itemize}
    
    
    \item[d)] Puedes elegir para este ejercicio una de las hipótesis de los ejercicios 1 y 2.
    
    \textbf{Solución:}
    
    Elección: Elipse del Ejercicio 2
    
    $$L_{\text{elipse}}(x, y, h, k, a, b, \delta) = \begin{cases}
    0 & \text{si } d \leq 1 \\
    \frac{d - 1}{\delta} & \text{si } 1 < d \leq 1 + \delta \\
    1 & \text{si } d > 1 + \delta
    \end{cases}$$
    
    donde $d = \sqrt{\frac{(x-h)^2}{a^2} + \frac{(y-k)^2}{b^2}}$
    
    
    \item[e)] Implementarás esto en la práctica asociada a este ejercicio.
    
    \textbf{Solución:}
    
    Implementación:
    
    \begin{verbatim}
def soft_loss_ellipse(x, y, h, k, a, b, delta):
    dist = np.sqrt((x - h)**2 / a**2 + (y - k)**2 / b**2)
    
    if dist <= 1:
        return 0
    elif dist <= 1 + delta:
        return (dist - 1) / delta
    else:
        return 1
    \end{verbatim}
    
\end{enumerate}
