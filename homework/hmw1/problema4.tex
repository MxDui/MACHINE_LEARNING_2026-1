\section*{4. [2.5 puntos] Función de pérdida relajada}

En los ejercicios 1 y 2 obtuviste funciones indicadoras para evaluar la pertenencia o no de puntos a hipótesis concretas.

\begin{enumerate}
    \item[a)] Debes extender la función indicadora para crear una nueva función que no evalúe de manera estrictamente ``dura'' la pertenencia a la hipótesis sino que permita considerar puntos cercanos a la frontera de la hipótesis de manera penalizada.
    
    \textbf{Solución:}
    
    Para extender la función indicadora binaria a una función suave (soft), podemos utilizar varias aproximaciones. Elegiremos una función lineal por partes basada en la distancia a la frontera:
    
    \textbf{Función de pérdida suave para elipse:}
    
    $$L_{\text{soft}}(x, y, \boldsymbol{\theta}, \delta) = \begin{cases}
    0 & \text{si } d(x,y) \leq 1 \text{ (dentro del elipse)} \\
    \frac{d(x,y) - 1}{\delta} & \text{si } 1 < d(x,y) \leq 1 + \delta \text{ (zona de transición)} \\
    1 & \text{si } d(x,y) > 1 + \delta \text{ (fuera de la zona)}
    \end{cases}$$
    
    donde:
    $$d(x,y) = \sqrt{\frac{(x-h)^2}{a^2} + \frac{(y-k)^2}{b^2}}$$
    
    es la distancia normalizada desde el punto $(x,y)$ al centro del elipse, y $\delta > 0$ es el parámetro que controla el ancho de la zona de transición.
    
    \textbf{Propiedades de esta función:}
    \begin{itemize}
        \item Es continua y diferenciable por partes
        \item Preserva la función indicadora original cuando $\delta \to 0$
        \item Permite penalización gradual cerca de la frontera
        \item $L_{\text{soft}} \in [0,1]$ para todo punto
    \end{itemize}
    
    
    \item[b)] Tú debes definir una manera de extender la frontera hasta cierta distancia máxima más allá de la frontera original. Por ejemplo: englobando a la hipótesis dentro de un círculo con cierto radio, o definiendo una distancia $\Delta d$ más allá de la frontera de la hipótesis.
    
    \textbf{Solución:}
    
    Para definir la extensión de la frontera, utilizaremos el concepto de \textbf{zona de transición} con ancho $\delta$:
    
    \textbf{Para una elipse con parámetros $\boldsymbol{\theta} = \{h, k, a, b\}$:}
    
    \begin{enumerate}
        \item \textbf{Frontera original:} Los puntos que satisfacen $d(x,y) = 1$
        
        \item \textbf{Zona de transición:} Los puntos que satisfacen $1 < d(x,y) \leq 1 + \delta$
        
        \item \textbf{Región exterior:} Los puntos que satisfacen $d(x,y) > 1 + \delta$
    \end{enumerate}
    
    donde $d(x,y) = \sqrt{\frac{(x-h)^2}{a^2} + \frac{(y-k)^2}{b^2}}$ es la distancia normalizada.
    
    \textbf{Interpretación geométrica:}
    
    La zona de transición forma una "cáscara elíptica" alrededor del elipse original. Los puntos en esta cáscara están entre:
    \begin{itemize}
        \item El elipse original: $\frac{(x-h)^2}{a^2} + \frac{(y-k)^2}{b^2} = 1$
        \item El elipse expandido: $\frac{(x-h)^2}{(a(1+\delta))^2} + \frac{(y-k)^2}{(b(1+\delta))^2} = 1$
    \end{itemize}
    
    \textbf{Elección del parámetro $\delta$:}
    \begin{itemize}
        \item $\delta$ pequeño ($\delta < 0.5$): Transición estrecha, más parecida a la función indicadora original
        \item $\delta$ mediano ($\delta \approx 1$): Balance entre suavidad y precisión
        \item $\delta$ grande ($\delta > 2$): Transición muy gradual, mayor tolerancia
    \end{itemize}
    
    
    \item[c)] En la región donde extiendas la frontera la función de pérdida debe penalizar en lugar de evaluar a cero.
    
    \textbf{Solución:}
    
    La función de pérdida relajada en la zona de transición se define como:
    
    $$L_{\text{relajada}}(x, y) = \begin{cases}
    0 & \text{si } d(x,y) \leq 1 \text{ (interior del elipse - sin pérdida)} \\
    \frac{d(x,y) - 1}{\delta} & \text{si } 1 < d(x,y) \leq 1 + \delta \text{ (zona de transición - penalización gradual)} \\
    1 & \text{si } d(x,y) > 1 + \delta \text{ (exterior - pérdida máxima)}
    \end{cases}$$
    
    \textbf{Justificación de la penalización:}
    
    \begin{itemize}
        \item \textbf{Región interior ($d \leq 1$):} Pérdida nula porque el punto está correctamente clasificado dentro de la hipótesis.
        
        \item \textbf{Zona de transición ($1 < d \leq 1 + \delta$):} Penalización lineal proporcional a qué tan lejos está el punto de la frontera original. Esta penalización:
        \begin{itemize}
            \item Comienza en 0 cuando $d = 1$ (en la frontera)
            \item Crece linealmente hasta 1 cuando $d = 1 + \delta$
            \item Permite diferenciabilidad para algoritmos de optimización
        \end{itemize}
        
        \item \textbf{Región exterior ($d > 1 + \delta$):} Pérdida máxima (1) porque el punto está claramente fuera de la hipótesis, incluso considerando la tolerancia.
    \end{itemize}
    
    \textbf{Ventajas de esta función de pérdida:}
    
    \begin{itemize}
        \item \textbf{Continuidad:} La función es continua en todos los puntos
        \item \textbf{Monotonicidad:} A mayor distancia de la frontera, mayor penalización
        \item \textbf{Acotamiento:} $L_{\text{relajada}} \in [0,1]$ siempre
        \item \textbf{Flexibilidad:} El parámetro $\delta$ controla la suavidad de la transición
    \end{itemize}
    
    
    \item[d)] Puedes elegir para este ejercicio una de las hipótesis de los ejercicios 1 y 2.
    
    \textbf{Solución:}
    
    \textbf{Elección: Hipótesis de elipse del Ejercicio 2}
    
    Elegimos trabajar con la clase de hipótesis de elipses $\mathcal{H}_{\text{eli}}$ por las siguientes razones:
    
    \begin{itemize}
        \item \textbf{Facilidad de cálculo:} La distancia normalizada $d(x,y) = \sqrt{\frac{(x-h)^2}{a^2} + \frac{(y-k)^2}{b^2}}$ es directa de calcular
        
        \item \textbf{Interpretación geométrica clara:} La zona de transición forma una cáscara elíptica uniforme
        
        \item \textbf{Diferenciabilidad:} La función es diferenciable en casi todas partes, útil para optimización
        
        \item \textbf{Parametrización simple:} Solo 4 parámetros $\{h, k, a, b\}$ más el parámetro de suavidad $\delta$
    \end{itemize}
    
    \textbf{Función de pérdida relajada completa para elipse:}
    
    $$L_{\text{elipse\_relajada}}(x, y, h, k, a, b, \delta) = \begin{cases}
    0 & \text{si } \sqrt{\frac{(x-h)^2}{a^2} + \frac{(y-k)^2}{b^2}} \leq 1 \\
    \frac{\sqrt{\frac{(x-h)^2}{a^2} + \frac{(y-k)^2}{b^2}} - 1}{\delta} & \text{si } 1 < \sqrt{\frac{(x-h)^2}{a^2} + \frac{(y-k)^2}{b^2}} \leq 1 + \delta \\
    1 & \text{si } \sqrt{\frac{(x-h)^2}{a^2} + \frac{(y-k)^2}{b^2}} > 1 + \delta
    \end{cases}$$
    
    Esta función extiende suavemente la función indicadora binaria del elipse, permitiendo una clasificación más robusta y diferenciable.
    
    
    \item[e)] Implementarás esto en la práctica asociada a este ejercicio.
    
    \textbf{Solución - Esquema de implementación:}
    
    La implementación en Python de la función de pérdida relajada para elipse sería:
    
    \begin{verbatim}
import numpy as np
import matplotlib.pyplot as plt

def soft_loss_ellipse(x, y, h, k, a, b, delta):
    """
    Funcion de perdida relajada para hipotesis de elipse
    
    Parametros:
    - (x, y): coordenadas del punto a evaluar
    - (h, k): centro del elipse
    - (a, b): semiejes del elipse
    - delta: ancho de la zona de transicion
    
    Retorna:
    - Perdida en [0,1]
    """
    # Calcular distancia normalizada
    dist = np.sqrt((x - h)**2 / a**2 + (y - k)**2 / b**2)
    
    if dist <= 1:
        return 0  # Interior: sin perdida
    elif dist <= 1 + delta:
        return (dist - 1) / delta  # Transicion suave
    else:
        return 1  # Exterior: perdida maxima

def visualize_soft_loss(h, k, a, b, delta):
    """Visualizar la funcion de perdida relajada"""
    x = np.linspace(h - 2*a, h + 2*a, 100)
    y = np.linspace(k - 2*b, k + 2*b, 100)
    X, Y = np.meshgrid(x, y)
    
    # Calcular perdida para cada punto
    Z = np.zeros_like(X)
    for i in range(len(x)):
        for j in range(len(y)):
            Z[j, i] = soft_loss_ellipse(X[j, i], Y[j, i], \
                                        h, k, a, b, delta)
    
    plt.contourf(X, Y, Z, levels=20, cmap='viridis')
    plt.colorbar(label='Perdida')
    plt.title(f'Funcion de perdida relajada (delta={delta})')
    plt.xlabel('x')
    plt.ylabel('y')
    plt.show()
    \end{verbatim}
    
\end{enumerate}
