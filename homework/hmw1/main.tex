\documentclass[12pt,a4paper]{article}
\usepackage[utf8]{inputenc}
\usepackage[T1]{fontenc}
\usepackage[spanish, activeacute, es-noindentfirst]{babel}
\usepackage{amsmath}
\usepackage{amsfonts}
\usepackage{amssymb}
\usepackage{geometry}
\usepackage{graphicx}
\usepackage{xcolor}
\usepackage{algorithm}
\usepackage{algpseudocode}
% Comment package removed - using % comments instead

% Define Greek letters and special symbols
\newcommand{\thetavec}{\boldsymbol{\theta}}
\newcommand{\alphavar}{\alpha}
\newcommand{\betavar}{\beta}
\newcommand{\deltavar}{\delta}
\newcommand{\epsilonvar}{\varepsilon}

% Indicator macro
\newcommand{\ind}[1]{\mathbb{1}_{\{#1\}}}

\geometry{margin=2.5cm}

\title{Tarea 1\\
\large Aprendizaje supervisado, funciones de pérdida y dimensión VC\\
Reconocimiento de Patrones y A.A. (semestre 2026-01)}
\author{David Rivera Morales}
\date{}

\begin{document}
\maketitle

\noindent\textbf{Instrucciones:} Resuelve los siguientes ejercicios de manera individual. La entrega es en un documento PDF. Puedes escribir tus respuestas en el documento mediante algún software o completamente a mano. Si lo haces a mano, cada paso debe ser legible, y las fotos que tomes se deben incorporar como parte del documento PDF.

\section*{1. [2.5 puntos] Dimensión VC}

Para una clase de hipótesis de triángulos $\mathcal{H}_{\text{tri}}$:

\begin{enumerate}
    \item[a)] Escribe una función indicadora para evaluar si un punto 2D pertenece o no a una hipótesis de un triángulo particular $h_{\boldsymbol{\theta}} \in \mathcal{H}_{\text{tri}}$.
    
    \textbf{Solución:}
    
    La función indicadora para determinar si un punto $(x,y)$ pertenece a un triángulo con vértices en $(x_1, y_1)$, $(x_2, y_2)$, $(x_3, y_3)$ se puede definir utilizando \textbf{coordenadas baricéntricas}:
    
    $$h_{\boldsymbol{\theta}}(x,y) = \begin{cases}
    1 & \text{si } \lambda_1 \geq 0, \lambda_2 \geq 0, \lambda_3 \geq 0 \\
    0 & \text{en caso contrario}
    \end{cases}$$
    
    donde las coordenadas baricéntricas $\lambda_1$, $\lambda_2$, $\lambda_3$ se calculan como:
    
    $$\lambda_1 = \frac{(y_2 - y_3)(x - x_3) + (x_3 - x_2)(y - y_3)}{(y_2 - y_3)(x_1 - x_3) + (x_3 - x_2)(y_1 - y_3)}$$
    
    $$\lambda_2 = \frac{(y_3 - y_1)(x - x_3) + (x_1 - x_3)(y - y_3)}{(y_2 - y_3)(x_1 - x_3) + (x_3 - x_2)(y_1 - y_3)}$$
    
    $$\lambda_3 = 1 - \lambda_1 - \lambda_2$$
    
    \textbf{Interpretación geométrica:} Un punto está dentro del triángulo si y solo si todas sus coordenadas baricéntricas son no negativas. Esto equivale a que el punto puede expresarse como una combinación convexa de los tres vértices.
    
    \textbf{Método alternativo (usando áreas):}
    
    $$h_{\boldsymbol{\theta}}(x,y) = \begin{cases}
    1 & \text{si } |A_{ABC}| = |A_{PBC}| + |A_{APC}| + |A_{ABP}| \\
    0 & \text{en caso contrario}
    \end{cases}$$
    
    donde $A_{ABC}$ es el área del triángulo con vértices $A=(x_1,y_1)$, $B=(x_2,y_2)$, $C=(x_3,y_3)$, y $P=(x,y)$ es el punto a evaluar.
    
    % HINT: La función indicadora debe verificar si un punto (x,y) está dentro del triángulo.
    % Puedes usar el método de coordenadas baricéntricas o el método del área.
    % 
    % SOLUCIÓN SUGERIDA:
    % - Parametriza el triángulo con 6 parámetros: θ = (x1, y1, x2, y2, x3, y3)
    % - Función indicadora usando áreas:
    %   h_θ(x,y) = 1 si el punto está dentro, 0 si está fuera
    % - Método del área: Un punto está dentro si:
    %   Area(ABC) = Area(PBC) + Area(APC) + Area(ABP)
    % - Alternativamente, usa coordenadas baricéntricas:
    %   El punto P está dentro si todas las coordenadas baricéntricas son >= 0
    
    \item[b)] Debes escribir la función parametrizada por los parámetros $\boldsymbol{\theta}$ que tú consideres.
    
    \textbf{Solución:}
    
    La parametrización natural para un triángulo en $\mathbb{R}^2$ requiere \textbf{6 parámetros}:
    
    $$\boldsymbol{\theta} = (x_1, y_1, x_2, y_2, x_3, y_3) \in \mathbb{R}^6$$
    
    donde $(x_i, y_i)$ representa las coordenadas del $i$-ésimo vértice del triángulo.
    
    La función parametrizada es entonces:
    
    $$h_{\boldsymbol{\theta}}: \mathbb{R}^2 \rightarrow \{0,1\}$$
    
    $$h_{\boldsymbol{\theta}}(x,y) = \mathbb{1}_{\{\text{punto } (x,y) \text{ está dentro del triángulo definido por } \boldsymbol{\theta}\}}$$
    
    \textbf{Expresión explícita:}
    
    $$h_{\boldsymbol{\theta}}(x,y) = \begin{cases}
    1 & \text{si } \begin{aligned}
    &\text{sgn}((x-x_1)(y_2-y_1) - (y-y_1)(x_2-x_1)) = \\
    &\text{sgn}((x-x_2)(y_3-y_2) - (y-y_2)(x_3-x_2)) = \\
    &\text{sgn}((x-x_3)(y_1-y_3) - (y-y_3)(x_1-x_3))
    \end{aligned} \\
    0 & \text{en caso contrario}
    \end{cases}$$
    
    donde $\text{sgn}$ es la función signo. Esta formulación verifica que el punto esté del mismo lado de las tres aristas del triángulo.
    
    \textbf{Implementación en Python:}
    
    \begin{verbatim}
def indicator_triangle(x, y, theta):
    x1, y1, x2, y2, x3, y3 = theta
    
    # Calcular el denominador común
    denominator = (y2 - y3) * (x1 - x3) + (x3 - x2) * (y1 - y3)
    
    # Si el denominador es cero, el triángulo es degenerado
    if abs(denominator) < 1e-10:
        return 0
    
    # Calcular coordenadas baricéntricas
    lambda1 = ((y2 - y3) * (x - x3) + (x3 - x2) * (y - y3)) / denominator
    lambda2 = ((y3 - y1) * (x - x3) + (x1 - x3) * (y - y3)) / denominator
    lambda3 = 1 - lambda1 - lambda2
    
    # Verificar si el punto está dentro del triángulo
    return 1 if (lambda1 >= 0 and lambda2 >= 0 and lambda3 >= 0) else 0
    \end{verbatim}
    
    \textbf{Observación:} La clase de hipótesis de triángulos es:
    
    $$\mathcal{H}_{\text{tri}} = \{h_{\boldsymbol{\theta}} : \boldsymbol{\theta} \in \mathbb{R}^6 \text{ define un triángulo no degenerado}\}$$
    
    Un triángulo es no degenerado si $|(y_2 - y_3)(x_1 - x_3) + (x_3 - x_2)(y_1 - y_3)| > 0$, es decir, los tres vértices no son colineales.
    
    % HINT: Los parámetros más naturales son los 3 vértices del triángulo.
    % theta = {(x1,y1), (x2,y2), (x3,y3)} - 6 parámetros en total
    % 
    % EJEMPLO DE FUNCIÓN:
    % def indicator_triangle(x, y, theta):
    %     x1, y1, x2, y2, x3, y3 = theta
    %     # Calcular usando el método de tu elección
    %     # Retornar 1 si está dentro, 0 si está fuera
    
    \item[c)] Luego usarás dicha función en la práctica asociada a este ejercicio para demostrar exhaustivamente (mediante un programa en Python) que la dimensión VC de la clase $\mathcal{H}_{\text{tri}}$ de triángulos es 7.
    
    \textbf{Solución:}
    
    Para demostrar que la dimensión VC de los triángulos es 7, se debe:
    
    \textbf{1. Demostrar que existen 7 puntos que pueden ser ``shattered'':}
    
    Una configuración que funciona es usar 6 puntos en los vértices de un hexágono regular y un punto en el centro:
    \begin{itemize}
        \item Puntos del hexágono: $(\cos(k\pi/3), \sin(k\pi/3))$ para $k = 0, 1, 2, 3, 4, 5$
        \item Punto central: $(0, 0)$
    \end{itemize}
    
    \textbf{2. Demostrar que NO existen 8 puntos que puedan ser shattered:}
    
    Para cualquier conjunto de 8 puntos en posición general, siempre existe al menos un etiquetado de las $2^8 = 256$ posibles formas que no puede ser realizado por ningún triángulo. Esto se debe a que:
    
    \begin{itemize}
        \item Un triángulo divide el plano en exactamente 2 regiones (interior y exterior)
        \item Con 8 puntos, existen etiquetados que requieren regiones no conexas para la clase positiva
        \item Por ejemplo, si 4 puntos forman un cuadrilátero convexo y los otros 4 están dentro, no se puede separar solo los puntos internos sin incluir algunos externos
    \end{itemize}
    
    \textbf{Pseudocódigo para la implementación:}
    \begin{verbatim}
    # Verificar que 7 puntos pueden ser shattered
    puntos_7 = generar_configuracion_hexagono_centro()
    for etiquetado in all_labelings(2^7):
        assert existe_triangulo_separador(puntos_7, etiquetado)
    
    # Verificar que 8 puntos NO pueden ser shattered
    puntos_8 = generar_8_puntos_posicion_general()
    contador_no_separables = 0
    for etiquetado in all_labelings(2^8):
        if not existe_triangulo_separador(puntos_8, etiquetado):
            contador_no_separables += 1
    assert contador_no_separables > 0
    \end{verbatim}
    
    Por tanto, la dimensión VC de la clase de hipótesis de triángulos es exactamente 7.
    
    % HINT: Para demostrar VC-dim = 7:
    % 1. Encuentra 7 puntos que pueden ser shattered (separados de todas las 2^7 formas posibles)
    % 2. Demuestra que NO existen 8 puntos que puedan ser shattered
    % 
    % ESTRATEGIA PYTHON:
    % - Genera conjuntos de 7 puntos y verifica que pueden ser shattered
    % - Para cada etiquetado posible (2^7 = 128), encuentra un triángulo que lo separe
    % - Prueba con conjuntos de 8 puntos y muestra que al menos un etiquetado no es separable
    % - Sugerencia: Usa puntos en configuración especial (e.g., hexágono regular + centro)
\end{enumerate}


\section*{2. [2.5 puntos] Clases de hipótesis}

Para una clase de hipótesis de elipses $\mathcal{H}_{\text{eli}}$ donde un elipse con centro en la coordenada $(h, k)$ se describe con la siguiente ecuación:

\begin{equation}
\frac{(x - h)^2}{a^2} + \frac{(y - k)^2}{b^2} = 1
\end{equation}

donde $a > 0$ y $b > 0$ son los parámetros para los semiejes del elipse.

\begin{enumerate}
    \item[a)] Usa esta ecuación para escribir una función indicadora para evaluar si un punto en 2D pertenece o no a una hipótesis de un elipse particular $h_{\boldsymbol{\theta}} \in \mathcal{H}_{\text{eli}}$, donde $\boldsymbol{\theta} = \{h, k, a, b\}$.
    
    \textbf{Solución:}
    
    La función indicadora para un elipse parametrizado por $\boldsymbol{\theta} = \{h, k, a, b\}$ se define como:
    
    $$h_{\boldsymbol{\theta}}(x,y) = \begin{cases}
    1 & \text{si } \frac{(x - h)^2}{a^2} + \frac{(y - k)^2}{b^2} \leq 1 \\
    0 & \text{en caso contrario}
    \end{cases}$$
    
    donde:
    \begin{itemize}
        \item $(h, k)$ es el centro del elipse
        \item $a$ es el semieje mayor (o menor) en la dirección $x$
        \item $b$ es el semieje mayor (o menor) en la dirección $y$
        \item $a, b > 0$
    \end{itemize}
    
    Un punto $(x,y)$ pertenece al interior del elipse (incluyendo la frontera) cuando la expresión normalizada es menor o igual a 1.
    
    
    \item[b)] Escribe a manera de pseudocódigo un algoritmo para encontrar los parámetros $\boldsymbol{\theta}' = \{h', k', a', b'\}$ del elipse más pequeño que sólo permite un error menor o igual al 10\% de clasificaciones incorrectas (para una de dos clases) del total de puntos $\mathcal{X} = \{(x_1, y_1), (x_2, y_2), \ldots, (x_n, y_n)\}$.
    
    \textbf{Solución:}
    
    \begin{verbatim}
    ALGORITMO: Encontrar_Elipse_Minimo_Con_Tolerancia
    
    ENTRADA: 
    - X: conjunto de puntos {(x_i, y_i)}
    - Y: etiquetas {+1, -1}
    - tolerancia: 0.1 (10% error máximo)
    
    1. Separar puntos por clase:
       X_pos = {puntos con etiqueta +1}
       X_neg = {puntos con etiqueta -1}
       n = |X|
       max_errores = floor(tolerancia * n)
    
    2. Inicializar elipse mínimo para puntos positivos:
       h' = media(x_i) para i en X_pos
       k' = media(y_i) para i en X_pos  
       a' = 2 * desviacion_estandar(x_i) para i en X_pos
       b' = 2 * desviacion_estandar(y_i) para i en X_pos
    
    3. Ajustar tamaño del elipse:
       factor_escala = 0.1
       MIENTRAS contar_errores(X, Y, {h', k', a', b'}) > max_errores:
           a' = a' * (1 + factor_escala)
           b' = b' * (1 + factor_escala)
    
    4. RETORNAR theta' = {h', k', a', b'}
    
    FUNCIÓN contar_errores(X, Y, theta):
        errores = 0
        PARA cada punto (x_i, y_i) con etiqueta y_i:
            prediccion = h_theta(x_i, y_i)
            etiqueta_esperada = 1 si y_i == +1, sino 0
            SI prediccion != etiqueta_esperada:
                errores += 1
        RETORNAR errores
    \end{verbatim}
    
    
    \item[c)] Ahora encuentra también a manera de pseudocódigo los parámetros del elipse más grande que no introduce nuevos errores. Puedes partir de los parámetros $\boldsymbol{\theta}'$ del inciso anterior.
    
    \textbf{Solución:}
    
    \begin{verbatim}
    ALGORITMO: Encontrar_Elipse_Maximo_Sin_Nuevos_Errores
    
    ENTRADA: 
    - X, Y: puntos y etiquetas
    - theta' = {h', k', a', b'}: parametros del elipse minimo del inciso b)
    
    1. Calcular errores actuales:
       errores_iniciales = contar_errores(X, Y, theta')
    
    2. Calcular distancia normalizada de puntos negativos al elipse:
       distancias_neg = []
       PARA cada punto (x_i, y_i) con etiqueta -1:
           dist_norm = sqrt((x_i - h')²/a'² + (y_i - k')²/b'²)
           SI dist_norm < 1:  # Punto negativo dentro del elipse
               distancias_neg.append(1 - dist_norm)
           SINO:
               distancias_neg.append(dist_norm - 1)
    
    3. Encontrar el factor de expansión máximo:
       factor_max = 1.0
       epsilon = 1e-6
       
       PARA cada punto negativo (x_i, y_i) FUERA del elipse:
           # Calcular factor máximo sin incluir este punto
           dist_norm = sqrt((x_i - h')²/a'² + (y_i - k')²/b'²)
           factor_candidato = 1 / dist_norm - epsilon
           factor_max = max(factor_max, factor_candidato)
    
    4. Expandir el elipse:
       a'' = a' * factor_max
       b'' = b' * factor_max
       theta'' = {h', k', a'', b''}
    
    5. Verificar que no se introduzcan nuevos errores:
       SI contar_errores(X, Y, theta'') > errores_iniciales:
           # Reducir factor de expansión iterativamente
           factor_max = factor_max * 0.99
           REPETIR paso 4 y 5
    
    6. RETORNAR theta'' = {h', k', a'', b''}
    \end{verbatim}
    
    
    \item[d)] Finalmente propón como hipótesis para la función indicadora un elipse que esté entre ambos (de los incisos b y c), por ejemplo a la mitad del camino paramétricamente hablando.
    
    \textbf{Solución:}
    
    La hipótesis final se obtiene mediante interpolación lineal entre los dos elipses:
    
    $$\boldsymbol{\theta}_{\text{final}} = \alpha \boldsymbol{\theta}' + (1-\alpha) \boldsymbol{\theta}''$$
    
    donde $\alpha = 0.5$ para obtener el punto medio.
    
    Específicamente:
    \begin{align}
    h_{\text{final}} &= h' \quad \text{(el centro no cambia)} \\
    k_{\text{final}} &= k' \quad \text{(el centro no cambia)} \\
    a_{\text{final}} &= \frac{a' + a''}{2} \\
    b_{\text{final}} &= \frac{b' + b''}{2}
    \end{align}
    
    Por tanto, la función indicadora final es:
    
    $$h_{\text{final}}(x,y) = \begin{cases}
    1 & \text{si } \frac{(x - h')^2}{a_{\text{final}}^2} + \frac{(y - k')^2}{b_{\text{final}}^2} \leq 1 \\
    0 & \text{en caso contrario}
    \end{cases}$$
    
    Esta elección balanceada permite un compromiso entre el elipse más pequeño (que minimiza el error de entrenamiento) y el elipse más grande (que maximiza el margen), lo que puede resultar en mejor generalización.
    
    
    \item[e)] Implementarás esto en la práctica asociada a este ejercicio.
\end{enumerate}



\section*{3. [2.5 puntos] Función de pérdida esperada}

Dados los valores de una función de pérdida $L_{kj}$, el riesgo esperado

\begin{equation}
E[L] = \sum_{k} \sum_{j} \int_{R_j} L_{kj} p(x, \mathcal{C}_k) dx
\end{equation}

se minimiza si para cada $x$ escogemos la clase para minimizar la siguiente expresión (ya que $p(x)$ no contiene al parámetro de clase):

\begin{equation}
\arg\min_{j} \left( \sum_{k} L_{kj} p(\mathcal{C}_k|x) p(x) \right) = \arg\min_{j} \left( \sum_{k} L_{kj} p(\mathcal{C}_k|x) \right)
\end{equation}

Verifica que si $L_{kj} = 1 - I_{kj}$, donde $I_{kj}$ son los elementos de la matriz identidad, entonces la minimización anterior se reduce a escoger la clase $i$ con la probabilidad más grande $p(\mathcal{C}_i|x)$.

\textbf{Solución:}

Con $L_{kj} = 1 - I_{kj}$:
$$L_{kj} = \begin{cases}
0 & \text{si } k = j \\
1 & \text{si } k \neq j
\end{cases}$$

La expresión a minimizar:
\begin{align}
\sum_{k} L_{kj} p(\mathcal{C}_k|x) &= \sum_{k} (1 - I_{kj}) p(\mathcal{C}_k|x) \\
&= \sum_{k} p(\mathcal{C}_k|x) - \sum_{k} I_{kj} p(\mathcal{C}_k|x) \\
&= 1 - p(\mathcal{C}_j|x)
\end{align}

Minimizar $[1 - p(\mathcal{C}_j|x)]$ equivale a maximizar $p(\mathcal{C}_j|x)$.

Por tanto, la regla óptima es:
$$i^* = \arg\max_{j} p(\mathcal{C}_j|x)$$




\section*{4. [2.5 puntos] Función de pérdida relajada}

En los ejercicios 1 y 2 obtuviste funciones indicadoras para evaluar la pertenencia o no de puntos a hipótesis concretas.

\begin{enumerate}
    \item[a)] Debes extender la función indicadora para crear una nueva función que no evalúe de manera estrictamente ``dura'' la pertenencia a la hipótesis sino que permita considerar puntos cercanos a la frontera de la hipótesis de manera penalizada.
    
    \textbf{Solución:}
    
    Función de pérdida suave para la elipse:
    
    $$L_{\text{soft}}(x, y) = \begin{cases}
    0 & \text{si } d(x,y) \leq 1 \\
    \frac{d(x,y) - 1}{\delta} & \text{si } 1 < d(x,y) \leq 1 + \delta \\
    1 & \text{si } d(x,y) > 1 + \delta
    \end{cases}$$
    
    donde $d(x,y) = \sqrt{\frac{(x-h)^2}{a^2} + \frac{(y-k)^2}{b^2}}$ y $\delta > 0$ controla el ancho de transición.
    
    
    \item[b)] Tú debes definir una manera de extender la frontera hasta cierta distancia máxima más allá de la frontera original. Por ejemplo: englobando a la hipótesis dentro de un círculo con cierto radio, o definiendo una distancia $\Delta d$ más allá de la frontera de la hipótesis.
    
    \textbf{Solución:}
    
    Zona de transición con ancho $\delta$:
    
    \begin{itemize}
        \item Frontera original: $d(x,y) = 1$
        \item Zona de transición: $1 < d(x,y) \leq 1 + \delta$
        \item Región exterior: $d(x,y) > 1 + \delta$
    \end{itemize}
    
    La zona forma una cáscara elíptica entre:
    \begin{itemize}
        \item Elipse original: $\frac{(x-h)^2}{a^2} + \frac{(y-k)^2}{b^2} = 1$
        \item Elipse expandida: $\frac{(x-h)^2}{(a(1+\delta))^2} + \frac{(y-k)^2}{(b(1+\delta))^2} = 1$
    \end{itemize}
    
    
    \item[c)] En la región donde extiendas la frontera la función de pérdida debe penalizar en lugar de evaluar a cero.
    
    \textbf{Solución:}
    
    $$L_{\text{relajada}}(x, y) = \begin{cases}
    0 & \text{si } d(x,y) \leq 1 \\
    \frac{d(x,y) - 1}{\delta} & \text{si } 1 < d(x,y) \leq 1 + \delta \\
    1 & \text{si } d(x,y) > 1 + \delta
    \end{cases}$$
    
    Penalización:
    \begin{itemize}
        \item Interior ($d \leq 1$): Sin pérdida
        \item Transición ($1 < d \leq 1 + \delta$): Penalización lineal de 0 a 1
        \item Exterior ($d > 1 + \delta$): Pérdida máxima
    \end{itemize}
    
    
    \item[d)] Puedes elegir para este ejercicio una de las hipótesis de los ejercicios 1 y 2.
    
    \textbf{Solución:}
    
    Elección: Elipse del Ejercicio 2
    
    $$L_{\text{elipse}}(x, y, h, k, a, b, \delta) = \begin{cases}
    0 & \text{si } d \leq 1 \\
    \frac{d - 1}{\delta} & \text{si } 1 < d \leq 1 + \delta \\
    1 & \text{si } d > 1 + \delta
    \end{cases}$$
    
    donde $d = \sqrt{\frac{(x-h)^2}{a^2} + \frac{(y-k)^2}{b^2}}$
    
    
    \item[e)] Implementarás esto en la práctica asociada a este ejercicio.
    
    \textbf{Solución:}
    
    Implementación:
    
    \begin{verbatim}
def soft_loss_ellipse(x, y, h, k, a, b, delta):
    dist = np.sqrt((x - h)**2 / a**2 + (y - k)**2 / b**2)
    
    if dist <= 1:
        return 0
    elif dist <= 1 + delta:
        return (dist - 1) / delta
    else:
        return 1
    \end{verbatim}
    
\end{enumerate}


\end{document}